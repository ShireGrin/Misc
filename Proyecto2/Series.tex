\documentclass{article}

\usepackage{amsmath}
\usepackage{blindtext}
\pagenumbering{gobble}

\title{Parte 3}
\begin{document}
\maketitle
\begin{center}
\section{Series}
\end{center}
\newpage

\subsection{Altura}

Sí el diámetro de una esfera es D, su altura está dada por:
\[
	H = D
\]
Sí a una esfera de diámetro D se le colocara otra esfera de diámetro $ \frac{3}{4}D $ encima, y luego otra esfera de $ (\frac{3}{4})^2D $, y se continúa hasta el infinito; la secuencia para la altura de todas estas esferas apiladas una sobre otra sería:
\[
	\Bigg\{D, \frac{3}{4}D, \Big(\frac{3}{4}\Big)^2D, \Big(\frac{3}{4}\Big)^3D, ...\Bigg\}
\]
La serie estaría dada por:
\[
	D\sum_{i=0}^{\infty}{\Big(\frac{3}{4}\Big)^{i}}
\]
Note que la serie es geométrica:
\[
	D\sum_{i=0}^{\infty}{\Big(\frac{3}{4}\Big)^{i}}
\]
Sabemos que las series geométricas convergen, y que su valor está dado por: $$ S = \frac{a}{1 - r} $$ se tiene que: $$ a = 1 ; r = \frac{3}{4}$$ Entonces: $$ S = \frac{1}{1-\frac{3}{4}} $$ $$ S = 4$$
Así que la altura de las esferas es: 
\[
	4(D)
\]
Si lo notamos como una función:
\[
	H(D) = 4D
\]
La suma de los carnés (201908274, 201901458) de los integrantes del grupo es: 63
Al evaluar la función de la altura en 63:
\[
	H(63) = 4*63
\]
\[
	H = 252
\]

\newpage
\subsection{Área Superficial}
El área superficial de una esfera en términos de su diámetro está dado por:
\[
	4\pi(\frac{D}{2})^2
\]
Sí a una esfera de diámetro D se le colocara otra esfera de diámetro $ \frac{3}{4}D $ encima, y luego otra esfera de $ (\frac{3}{4})^2D $, y se continúa hasta el infinito; la secuencia para el área superficial sería:
\[
	\Bigg\{4\pi(\frac{D}{2})^2, 4\pi\Big(\frac{3}{4}*\frac{D}{2}\Big)^2, 4\pi\Big(\Big(\frac{3}{4}\Big)^2*\frac{D}{2}\Big)^2, 4\pi\Big(\Big(\frac{3}{4}\Big)^3*\frac{D}{2}\Big)^2, ... \Bigg\}
\]
La serie estaría dada por:
\[
	4\pi\Big(\frac{D}{2}\Big)^2\sum_{i=0}^{\infty}{\Big(\frac{3}{4}\Big)^{2i}}
\]
Note que la serie es geométrica:
\[
	\pi D^2\sum_{i=0}^{\infty}{\Big(\frac{9}{16}\Big)^{i}}
\]
Sabemos que las series geométricas convergen, y que su valor está dado por: $$ S = \frac{a}{1 - r} $$ se tiene que: $$ a = 1 ; r = \frac{9}{16}$$ Entonces: $$ S = \frac{1}{1-\frac{9}{16}} $$ $$ S = \frac{16}{7}$$
Así que el área superficial de las esferas es: 
\[
	\frac{16\pi}{7}(D)^2
\]
Si lo notamos como una función:
\[
	S(D) = \frac{16\pi}{7}(D)^2
\]
La suma de los carnés (201908274, 201901458) de los integrantes del grupo es: 63
Al evaluar la función del área superficial en 63:
\[
	S(63) = \frac{16\pi}{9}(63)^2
\]
\[
	S = 28500.52855
\]

\newpage
\subsection{Volúmen}
El volumen de una esfera en términos de su diámetro está dado por:
\[
	\frac{4}{3}\pi(\frac{D}{2})^3
\]
Sí a una esfera de diámetro D se le colocara otra esfera de diámetro $ \frac{3}{4}D $ encima, y luego otra esfera de $ (\frac{3}{4})^2D $, y se continúa hasta el infinito; la secuencia para el volumen sería:
\[
	\Bigg\{\frac{4}{3}\pi(\frac{D}{2})^3, \frac{4}{3}\pi\Big(\frac{3}{4}*\frac{D}{2}\Big)^3, \frac{4}{3}\pi\Big(\Big(\frac{3}{4}\Big)^2*\frac{D}{2}\Big)^3, \frac{4}{3}\pi\Big(\Big(\frac{3}{4}\Big)^3*\frac{D}{2}\Big)^3, ... \Bigg\}
\]
La serie estaría dada por:
\[
	\frac{4\pi}{3}\Big(\frac{D}{2}\Big)^3\sum_{i=0}^{\infty}{\Big(\frac{3}{4}\Big)^{3i}}
\]
Note que la serie es geométrica, y su primer valor (cuando i=0) es 1.
Se puede simplicar la serie, así:
\[
	\frac{\pi}{6}D^3\sum_{i=0}^{\infty}{\Big(\frac{27}{64}\Big)^{i}}
\]
Sabemos que las series geométricas convergen, y que su valor está dado por: $$ S = \frac{a}{1 - r} $$ se tiene que: $$ a = 1 ; r = \frac{27}{64}$$ Entonces: $$ S = \frac{1}{1-\frac{27}{64}} $$ $$ S = \frac{64}{27}$$
Así que el volumen de las esferas es: 
\[
	\frac{64\pi}{37*6}(D)^3
\]
Si lo notamos como una función:
\[
	V(D) = \frac{32\pi}{111}(D)^3
\]
La suma de los carnés (201908274, 201901458) de los integrantes del grupo es: 63
Al evaluar la función del volumen en 63:
\[
	V(63) = \frac{32\pi}{111}(63)^3
\]
\[
	V = 226463.6593
\]
\newpage

\section{Referencias}
James Stewart, (2018) Cálculo: Trascendentes Tempranas, Ciudad de México, CENGAGE\\
(2014), JCommon, 25 de Octubre, de JCommon, sitio: http://www.jfree.org/jcommon/\\
(2019), Calculadora Polar Desmos, 23 de Octubre sitio\\
https://www.desmos.com/calculator/ms3eghkkgz\\
Documentación de numpy: https://docs.scipy.org/doc/numpy/reference/\\
Documentación de matplotlib: https://matplotlib.org/3.1.1/contents.html\\
El siguiente repositorio contiene el código utilizado para realizar ciertas partes del proyecto. (La organización de dichas partes se describe en el repositorio):\\https://github.com/ShireGrin/Misc/tree/master/Proyecto2
\end{document}
